% Options for packages loaded elsewhere
\PassOptionsToPackage{unicode}{hyperref}
\PassOptionsToPackage{hyphens}{url}
%
\documentclass[
]{article}
\usepackage{amsmath,amssymb}
\usepackage{iftex}
\ifPDFTeX
  \usepackage[T1]{fontenc}
  \usepackage[utf8]{inputenc}
  \usepackage{textcomp} % provide euro and other symbols
\else % if luatex or xetex
  \usepackage{unicode-math} % this also loads fontspec
  \defaultfontfeatures{Scale=MatchLowercase}
  \defaultfontfeatures[\rmfamily]{Ligatures=TeX,Scale=1}
\fi
\usepackage{lmodern}
\ifPDFTeX\else
  % xetex/luatex font selection
\fi
% Use upquote if available, for straight quotes in verbatim environments
\IfFileExists{upquote.sty}{\usepackage{upquote}}{}
\IfFileExists{microtype.sty}{% use microtype if available
  \usepackage[]{microtype}
  \UseMicrotypeSet[protrusion]{basicmath} % disable protrusion for tt fonts
}{}
\makeatletter
\@ifundefined{KOMAClassName}{% if non-KOMA class
  \IfFileExists{parskip.sty}{%
    \usepackage{parskip}
  }{% else
    \setlength{\parindent}{0pt}
    \setlength{\parskip}{6pt plus 2pt minus 1pt}}
}{% if KOMA class
  \KOMAoptions{parskip=half}}
\makeatother
\usepackage{xcolor}
\usepackage[margin=1in]{geometry}
\usepackage{graphicx}
\makeatletter
\def\maxwidth{\ifdim\Gin@nat@width>\linewidth\linewidth\else\Gin@nat@width\fi}
\def\maxheight{\ifdim\Gin@nat@height>\textheight\textheight\else\Gin@nat@height\fi}
\makeatother
% Scale images if necessary, so that they will not overflow the page
% margins by default, and it is still possible to overwrite the defaults
% using explicit options in \includegraphics[width, height, ...]{}
\setkeys{Gin}{width=\maxwidth,height=\maxheight,keepaspectratio}
% Set default figure placement to htbp
\makeatletter
\def\fps@figure{htbp}
\makeatother
\setlength{\emergencystretch}{3em} % prevent overfull lines
\providecommand{\tightlist}{%
  \setlength{\itemsep}{0pt}\setlength{\parskip}{0pt}}
\setcounter{secnumdepth}{-\maxdimen} % remove section numbering
\usepackage{multirow}
\usepackage{diagbox}
\usepackage{booktabs}
\usepackage{longtable}
\usepackage{array}
\usepackage{multirow}
\usepackage{wrapfig}
\usepackage{float}
\usepackage{colortbl}
\usepackage{pdflscape}
\usepackage{tabu}
\usepackage{threeparttable}
\usepackage{threeparttablex}
\usepackage{booktabs}
\usepackage{longtable}
\usepackage{array}
\usepackage{multirow}
\usepackage{wrapfig}
\usepackage{float}
\usepackage{colortbl}
\usepackage{pdflscape}
\usepackage{tabu}
\usepackage{threeparttable}
\usepackage{threeparttablex}
\usepackage[normalem]{ulem}
\usepackage{makecell}
\usepackage{xcolor}
\ifLuaTeX
  \usepackage{selnolig}  % disable illegal ligatures
\fi
\IfFileExists{bookmark.sty}{\usepackage{bookmark}}{\usepackage{hyperref}}
\IfFileExists{xurl.sty}{\usepackage{xurl}}{} % add URL line breaks if available
\urlstyle{same}
\hypersetup{
  pdftitle={Supplementary Material},
  pdfauthor={Kevin Jin and Shu-Min Liao},
  hidelinks,
  pdfcreator={LaTeX via pandoc}}

\title{Supplementary Material}
\author{Kevin Jin and Shu-Min Liao}
\date{}

\begin{document}
\maketitle

\section{Simulations}\label{simulations}

\subsection{Simulation for Ordinal Response Variables with Multiple
Levels}\label{simulation-for-ordinal-response-variables-with-multiple-levels}

Let's now use simulations to check the efficacy this proposed method for
visualizing the dependence structure and interactions of categorical
data with an ordinal response which has multiple levels. We begin with
visualizing interactions of categorical data for ordinal responses with
multiple levels. We will begin with the cases of having three variables
(response variable \(Y\), explanatory variable \(X\), and grouping
variable \(G\)) and consider examples with more than three variables at
the end of this section.

In our simulations with 3 variables, we will consider a hypothetical
data set where the ordinal response \(Y\) has five categories
(\(y_1,\dots,y_5\)), ordinal predictor \(X\) has three categories
(\(x_1,x_2,x_3\)), and nominal predictor \(G\) has two categories
(\(g_1, g_2\)) unless otherwise specified. While these number of
categories were arbitrarily chosen, these combination of category number
for each variable will be sufficient to consider multiple different
relationships between the variables. Further, our BECCR predictions will
always be done with a bootstrap resampling of \(B=1000\) to ensure
consistency\footnote{We chose \(B=1000\) for the sake of efficiency and
  saving time because of additional simulations performed but not
  reported in this thesis.}.

\subsubsection{Simulating Hypothetical Data with Linear
Relationship}\label{simulating-hypothetical-data-with-linear-relationship}

For our first simulation, consider the following aggregated hypothetical
data set shown in Figures \ref{fig:mp_sim_SLR_ungroup} (mosaic plot) and
\ref{fig:BECCR_sim_SLR_ungroup} (BECCR Prediction Bubble Plot). We will
consider this aggregate visualization first, as we want to see what kind
of ``Simpson's Paradox'' may occur before accounting for grouping
variable \(G\). This will allow us to gain a better insight on how the
BECCR Prediction Bubble Plots can be used to potentially visualize these
amalgamation effects.

\begin{figure}
\centering
\includegraphics{Supplementary-Material_files/figure-latex/unnamed-chunk-2-1.pdf}
\caption{\label{fig:mp_sim_SLR_ungroup}Mosaic Plot of Aggregated
Hypothetical Data Set 1 without accounting for \(G\)}
\end{figure}

\begin{figure}
\centering
\includegraphics{Supplementary-Material_files/figure-latex/unnamed-chunk-4-1.pdf}
\caption{\label{fig:BECCR_sim_SLR_ungroup}BECCR Prediction Bubble Plot
of Aggregated Hypothetical Data Set 1 without accounting for \(G\)}
\end{figure}

Looking at these visualizations, both the BECCR Prediction Bubble Plot
and the mosaic plot do not seem to indicate that there is an
relationship between \(X\) and \(Y\). The hypothetical data set used is
displayed in Table \ref{tab:sim_SLR}. From this, we can see, that after
accounting for \(G\), there appears to be a linear relationship between
\(X\) and \(Y\), with it being negative or positive depending on the
level of \(G\). In Figures \ref{fig:mp_sim_SLR} and
\ref{fig:BECCR_sim_SLR}, we can see the mosaic plot and the BECCR
Prediction Bubble Plot of the data, after considering the third variable
\(G\). Looking at these figures, we can see that both the mosaic plot
and the BECCR Prediction Bubble Plot are able to capture the linear
relationship between \(X\) and \(Y\) after accounting for \(G\), with
the BECCR Prediction Bubble Plots displaying this linear pattern in a
much clearer fashion.

\begin{table}[H]

\caption{\label{tab:sim_SLR}Hypothetical Ordinal Data
                    with Linear Relationship Between $X$ and $Y$}
\centering
\begin{tabular}[t]{ccccccc}
\toprule
\diagbox{$Y$}{$(G,X)$} & $(g_1, x_1)$ & $(g_1, x_2)$ & $(g_1, x_3)$ & $(g_2, x_1)$ & $(g_2, x_2)$ & $(g_2, x_3)$\\
\midrule
$y_1$ & 100 & 1 & 1 & 1 & 1 & 100\\
$y_2$ & 1 & 1 & 1 & 1 & 1 & 1\\
$y_3$ & 1 & 100 & 1 & 1 & 100 & 1\\
$y_4$ & 1 & 1 & 1 & 1 & 1 & 1\\
$y_5$ & 1 & 1 & 100 & 100 & 1 & 1\\
\bottomrule
\end{tabular}
\end{table}

\begin{figure}
\centering
\includegraphics{Supplementary-Material_files/figure-latex/unnamed-chunk-6-1.pdf}
\caption{\label{fig:mp_sim_SLR}Mosaic Plot for Hypothetical Data with
Linear Relationships after Accounting for \(G\)}
\end{figure}

\begin{figure}
\centering
\includegraphics{Supplementary-Material_files/figure-latex/unnamed-chunk-7-1.pdf}
\caption{\label{fig:BECCR_sim_SLR}BECCR Prediction Bubble Plots for
Hypothetical Data with Linear Relationships after Accounting for \(G\)}
\end{figure}

\subsubsection{Simulating Hypothetical Data with Quadratic
Relationship}\label{simulating-hypothetical-data-with-quadratic-relationship}

Moving onto the next case, we want to determine how well BECCR
Prediction Bubble Plots do to visualize the following hypothetical
health data oulined in Table \ref{tab:sim_quadratic}. Without
considering \(G\), both the mosaic plot and the BECCR Prediction Bubble
Plot (seen in Figures \ref{fig:mp_sim_quadratic_ungroup} and
\ref{fig:BECCR_sim_quadratic_ungroup} respectively) seem to indicate no
relationship between \(X\) and \(Y\), which deviates what we observed
from the data in Table \ref{tab:sim_quadratic}. On the other hand, when
looking at Figure \ref{fig:mp_sim_quadratic} and Figure
\ref{fig:BECCR_sim_quadratic}, one can see a quadratic relationship
between \(X\) and \(Y\) within each level of the BECCR Prediction Bubble
Plot, but not so easily in the mosaic plot.

It is important to notice that the BECCR Bubble Prediction Plot seems to
``condense'' the results, and this is due to how the (checkerboard
copula) regression works, as well as the fact we are ``predicting'' a
response category. As a result, the quadratic relationship of \(X\) and
\(Y\) is better captured and displayed in the BECCR Prediction Bubble
Plot (Figure \ref{fig:BECCR_sim_quadratic}), while some people might
have difficulties visualizing this quadratic relationship in the mosaic
plot (Figure \ref{fig:mp_sim_quadratic}).

\begin{table}[H]

\caption{\label{tab:sim_quadratic}Hypothetical Ordinal Data
                    with Quadratic Relationship Between $X$ and $Y$}
\centering
\begin{tabular}[t]{ccccccc}
\toprule
\diagbox{$Y$}{$(G,X)$} & $(g_1, x_1)$ & $(g_1, x_2)$ & $(g_1, x_3)$ & $(g_2, x_1)$ & $(g_2, x_2)$ & $(g_2, x_3)$\\
\midrule
$y_1$ & 0 & 60 & 0 & 30 & 0 & 30\\
$y_2$ & 15 & 30 & 15 & 25 & 10 & 25\\
$y_3$ & 20 & 20 & 20 & 20 & 20 & 20\\
$y_4$ & 25 & 10 & 25 & 15 & 30 & 15\\
$y_5$ & 30 & 0 & 30 & 0 & 60 & 0\\
\bottomrule
\end{tabular}
\end{table}

\begin{figure}
\centering
\includegraphics{Supplementary-Material_files/figure-latex/unnamed-chunk-11-1.pdf}
\caption{\label{fig:mp_sim_quadratic_ungroup}Mosaic Plot of Hypothetical
Ordinal Data with Quadratic Relationship without Accounting for \(G\)}
\end{figure}

\begin{figure}
\centering
\includegraphics{Supplementary-Material_files/figure-latex/unnamed-chunk-13-1.pdf}
\caption{\label{fig:BECCR_sim_quadratic_ungroup}BECCR Prediction Bubble
Plots of Hypothetical Ordinal Data with Quadratic Relationship without
accounting for \(G\)}
\end{figure}

\begin{figure}
\centering
\includegraphics{Supplementary-Material_files/figure-latex/unnamed-chunk-14-1.pdf}
\caption{\label{fig:mp_sim_quadratic}Mosaic Plot for Hypothetical
Ordinal Data with Quadratic Relationship between \(X\) and \(Y\) after
accounting for \(G\)}
\end{figure}

\begin{figure}
\centering
\includegraphics{Supplementary-Material_files/figure-latex/unnamed-chunk-15-1.pdf}
\caption{\label{fig:BECCR_sim_quadratic}BECCR Prediction Bubble Plots
for Hypothetical Ordinal Data with Quadratic Relationship after
accounting for \(G\)}
\end{figure}

\subsubsection{Simulating Hypothetical Ordinal Data with a
Pattern}\label{simulating-hypothetical-ordinal-data-with-a-pattern}

For our third simulation, we want to see how well BECCR Prediction
Bubble Plots can do to visualize data with different patterns between
\(X\) and \(Y\) for different levels of \(G\). The hypothetical data set
for this case is outlined in Table \ref{tab:sim_pattern}, which shows an
irregular relationship between \(X\) and \(Y\) for \(G=g_1\), but there
still seems to be a pattern that exists.

Again, when not considering \(G\), the visualizations seem to indicate
no relationship between \(X\) and \(Y\) for both the mosaic plot and the
BECCR Prediction Bubble Plot (seen in Figures
\ref{fig:mp_sim_pattern_ungroup} and \ref{fig:BECCR_sim_pattern_ungroup}
respectively). It may be worth noting that in the BECCR Prediction
Bubble Plot (in Figure \ref{fig:BECCR_sim_pattern_ungroup}), some
aspects of the pattern relationship seems visible, and may suggest that
BECCR Prediction Bubble Plots applied to aggregated data without the
grouping variable \(G\) can still potentially offer some insights to the
relationships of the variables.

\begin{table}[H]

\caption{\label{tab:sim_pattern}Hypothetical Ordinal Data
                    with a Relationship Pattern Between $X$ and $Y$}
\centering
\begin{tabular}[t]{ccccccc}
\toprule
\diagbox{$Y$}{$(G,X)$} & $(g_1, x_1)$ & $(g_1, x_2)$ & $(g_1, x_3)$ & $(g_2, x_1)$ & $(g_2, x_2)$ & $(g_2, x_3)$\\
\midrule
$y_1$ & 50 & 50 & 1 & 1 & 1 & 150\\
$y_2$ & 1 & 1 & 1 & 1 & 1 & 1\\
$y_3$ & 50 & 1 & 50 & 1 & 150 & 1\\
$y_4$ & 1 & 1 & 1 & 1 & 1 & 1\\
$y_5$ & 1 & 50 & 50 & 150 & 1 & 1\\
\bottomrule
\end{tabular}
\end{table}

\begin{figure}
\centering
\includegraphics{Supplementary-Material_files/figure-latex/unnamed-chunk-18-1.pdf}
\caption{\label{fig:mp_sim_pattern_ungroup}Mosaic Plot of Hypothetical
Ordinal Data with a Pattern Relationship, aggregated without considering
\(G\).}
\end{figure}

\begin{figure}
\centering
\includegraphics{Supplementary-Material_files/figure-latex/unnamed-chunk-20-1.pdf}
\caption{\label{fig:BECCR_sim_pattern_ungroup}BECCR Prediction Bubble
Plot of Hypothetical Ordinal Data with a Pattern Relationship,
Aggregated without considering \(G\)}
\end{figure}

Looking at the mosaic plot in Figure \ref{fig:mp_sim_pattern}, we can
clearly see that the mosaic plot is able to capture the different
relationships between \(X\) and \(Y\) for \(G=g_1\) for different levels
of \(G\). Similarly, the BECCR Prediction Bubble Plot in Figure
\ref{fig:BECCR_sim_pattern} is also able to capture the relationship
between \(X\) and \(Y\), depending on \(G\), again in a much cleaner
manner. While both plots successfully capture the peculiar pattern of
\(X\) and \(Y\)'s relationship, we begin to see some differences between
the two plots. In particular, the part of the mosaic plot for \(g_1\)
was shorter than the other part of the plot for \(g_2\), which arises
due to the different sizes of the two subgroups. In the previous
simulations, each subgroup \(g_1\) and \(g_2\) have equal sizes, but
when the proportions change, the mosaic plots will change with it. While
this does not count as a drawback of mosaic plots, this observation
motivates us to simulate the next case.

\begin{figure}
\centering
\includegraphics{Supplementary-Material_files/figure-latex/unnamed-chunk-21-1.pdf}
\caption{\label{fig:mp_sim_pattern}Mosaic Plot for Hypothetical Ordinal
Data with Pattern After Accounting for \(G\)}
\end{figure}

\begin{figure}
\centering
\includegraphics{Supplementary-Material_files/figure-latex/unnamed-chunk-22-1.pdf}
\caption{\label{fig:BECCR_sim_pattern}BECCR Prediction Bubble Plot of
Hypothetical Ordinal Data with Pattern After Accounting for \(G\)}
\end{figure}

\subsubsection{Simulating Hypothetical Ordinal Data with Subgroups of
Differing
Sizes}\label{simulating-hypothetical-ordinal-data-with-subgroups-of-differing-sizes}

As mentioned, this simulation will be performed on a hypothetical data
set that will have \(g_1\) be in much smaller proportion than \(g_2\),
as we want to see how well BECCR Prediction Bubble Plots can help us
visualize the interactions of it. This data set can be seen in Table
\ref{tab:sim_prop}. Clearly, the proportion of observations in \(g_1\)
is much lower than those of \(g_2\). We can see this exhibited as the
relationship in \(g_2\) seems to dominate the visualizations in Figures
\ref{fig:mp_sim_prop_ungroup} and \ref{fig:BECCR_sim_prop_ungroup}.

\begin{table}[H]

\caption{\label{tab:sim_prop}Hypothetical Ordinal Data
                    with Differing Proportions of Subpopulations in $g_1$ and $g_2$}
\centering
\begin{tabular}[t]{ccccccc}
\toprule
\diagbox{$Y$}{$(G,X)$} & $(g_1, x_1)$ & $(g_1, x_2)$ & $(g_1, x_3)$ & $(g_2, x_1)$ & $(g_2, x_2)$ & $(g_2, x_3)$\\
\midrule
$y_1$ & 0 & 10 & 0 & 100 & 0 & 100\\
$y_2$ & 5 & 0 & 5 & 100 & 0 & 100\\
$y_3$ & 5 & 0 & 5 & 100 & 0 & 100\\
$y_4$ & 5 & 0 & 5 & 100 & 0 & 100\\
$y_5$ & 5 & 0 & 5 & 0 & 200 & 0\\
\bottomrule
\end{tabular}
\end{table}

\begin{figure}
\centering
\includegraphics{Supplementary-Material_files/figure-latex/unnamed-chunk-25-1.pdf}
\caption{\label{fig:mp_sim_prop_ungroup}Mosaic Plot of Ordinal Data with
Differing Proportions without accounting for \(G\)}
\end{figure}

\begin{figure}
\centering
\includegraphics{Supplementary-Material_files/figure-latex/unnamed-chunk-27-1.pdf}
\caption{\label{fig:BECCR_sim_prop_ungroup}BECCR Prediction Bubble Plot
of Ordinal Data with Differing Proportions without Accounting for \(G\)}
\end{figure}

We now want to see how the visualizations of the data between the mosaic
plot and the BECCR Prediction Bubble Plot changes after accounting for
\(G\). In Figure \ref{fig:mp_sim_prop}, we can clearly see that the
mosaic plot does capture the quadratic relationship inherent in the
data. However, the difference between the sizes of \(g_1\) and \(g_2\)
is extremely prominent. If the relationship was not as easy to
visualize, we can imagine that it will be hard to distinguish possible
relationships of \(X\) and \(Y\) for \(G=g_1\). On the other hand, the
BECCR Prediction Bubble Plot in Figure \ref{fig:BECCR_sim_prop} not only
captures the quadratic relationship between \(X\) and \(Y\), but also
helps us visualize the different relationships between \(X\) and \(Y\)
depending on what \(G\) is. While the data does get ``condensed'' again,
the overall quadratic relationship is depicted much more clearly than it
does in the mosaic plot in Figure \ref{fig:mp_sim_prop}. We expect that
these differences will be further exacerbated in a multi-dimensional
example with more than 3 variables, as we will see in the next
simulation case.

\begin{figure}
\centering
\includegraphics{Supplementary-Material_files/figure-latex/unnamed-chunk-28-1.pdf}
\caption{\label{fig:mp_sim_prop}Mosaic Plot for Hypothetical Ordinal
Data with Differing Proportions after Accounting for \(G\).}
\end{figure}

\begin{figure}
\centering
\includegraphics{Supplementary-Material_files/figure-latex/unnamed-chunk-29-1.pdf}
\caption{\label{fig:BECCR_sim_prop}BECCR Prediction Bubble Plot of
Hypothetical Ordinal Data with Differing Proportions after Accounting
for \(G\)}
\end{figure}

\subsubsection{Simulating Hypothetical multi-Dimensional Ordinal
Data}\label{simulating-hypothetical-multi-dimensional-ordinal-data}

\begin{table}[H]

\caption{\label{tab:sim_multi_z1}Hypothetical Ordinal Data when $Z = z_1$, }
\centering
\begin{tabular}[t]{ccccccc}
\toprule
\diagbox{$Y$}{$(G,X)$} & $(g_1, x_1)$ & $(g_1, x_2)$ & $(g_1, x_3)$ & $(g_2, x_1)$ & $(g_2, x_2)$ & $(g_2, x_3)$\\
\midrule
$y_1$ & 0 & 10 & 0 & 150 & 0 & 150\\
$y_2$ & 5 & 0 & 5 & 150 & 0 & 150\\
$y_3$ & 5 & 0 & 5 & 150 & 0 & 150\\
$y_4$ & 5 & 0 & 5 & 150 & 0 & 150\\
$y_5$ & 5 & 0 & 5 & 0 & 250 & 0\\
\bottomrule
\end{tabular}
\end{table}

\begin{table}[H]

\caption{\label{tab:sim_multi_z2}Hypothetical Ordinal Data when $Z=z_2$}
\centering
\begin{tabular}[t]{ccccccc}
\toprule
\diagbox{$Y$}{$(G,X)$} & $(g_1, x_1)$ & $(g_1, x_2)$ & $(g_1, x_3)$ & $(g_2, x_1)$ & $(g_2, x_2)$ & $(g_2, x_3)$\\
\midrule
$y_1$ & 150 & 1 & 1 & 3 & 3 & 10\\
$y_2$ & 1 & 1 & 1 & 3 & 3 & 3\\
$y_3$ & 1 & 150 & 1 & 3 & 10 & 3\\
$y_4$ & 1 & 1 & 1 & 3 & 3 & 3\\
$y_5$ & 1 & 1 & 150 & 10 & 3 & 3\\
\bottomrule
\end{tabular}
\end{table}

For this simulation, we are going deal with a similar data set as in the
prior simulations, but with an additional explanatory binary predictor
\(Z = \{z_1,z_2\}\). As we can see in Tables \ref{tab:sim_multi_z1} and
\ref{tab:sim_multi_z2} (tables for \(Z=z_1\) and \(Z=z_2\)
respectively), a combination of all the prior hypothetical data is
exhibited. Without considering both \(G\) and \(Z\) the aggregated plots
are shown in Figures \ref{fig:mp_sim_multi_ungroup} and
\ref{fig:BECCR_sim_multi_ungroup} for the mosaic plot and the BECCR
Prediction Bubble Plot respectively. From the BECCR Prediction Bubble
Plot, we can clearly see that there seems to have a discernible
association between \(Y\) and \(X\), without accounting for the other
variables, \(G\) and \(Z\).

\begin{figure}
\centering
\includegraphics{Supplementary-Material_files/figure-latex/unnamed-chunk-34-1.pdf}
\caption{\label{fig:mp_sim_multi_ungroup}Mosaic Plot of Hypothetical
Multi-Dimensional Aggregated Ordinal Data without accounting for \(G\)
and \(Z\).}
\end{figure}

\begin{figure}
\centering
\includegraphics{Supplementary-Material_files/figure-latex/unnamed-chunk-35-1.pdf}
\caption{\label{fig:BECCR_sim_multi_ungroup}BECCR Prediction Bubble Plot
of Hypothetical Multi-Dimensional Aggregated Ordinal Data without
accounting for \(G\) and \(Z\).}
\end{figure}

\begin{figure}
\centering
\includegraphics{Supplementary-Material_files/figure-latex/unnamed-chunk-37-1.pdf}
\caption{\label{fig:BECCR_sim_multi_noz}BECCR Prediction Bubble Plot for
Hypothetical Multivaraite Ordinal Data without Accounting for \(Z\)}
\end{figure}

How would these plots change after adding \(G\) and \(Z\) to the
checkerboard copula regression? We are particularly interested in how
effective BECCR Prediction Bubble Plots could capture and identify the
potential interactions between 3 predictors compared to the mosaic
plots. These visualizations are shown in Figures \ref{fig:mp_sim_multi}
and \ref{fig:BECCR_sim_multi}. Notice that the top plot in Figure
\ref{fig:BECCR_sim_multi} is for \(Z=z_1\), and the bottom plot is for
\(Z=z_2\). Immediately, we can see that the mosaic plot (Figure
\ref{fig:mp_sim_multi}) is really convoluted to look at, difficult to
interpret, and provides little useful insights about the dependence
structure of the variables.

On the contrary, the BECCR Prediction Bubble Plot (Figure
\ref{fig:BECCR_sim_multi}) clearly shows that for each combination of
variables \((Z,G,X)\), there is a noticeable relationship with \(Y\).
When \(Z=z_1\), if \(G=g_1\), there seems to be a ``negative'' quadratic
relationship between \(X\) and \(Y\); if \(G=g_2\), there instead seems
to be a positive quadratic relationship. Further, when \(Z=z_2\), if
\(G=g_1\), there seems to be a strong negative linear relationship
between \(X\) and \(Y\); if \(G=g_2\), there seems to be a moderate
positive relationship between \(X\) and \(Y\).

Comparing these results to the data in Table \ref{tab:sim_multi_z1} and
\ref{tab:sim_multi_z2}, we can see the BECCR Prediction Bubble Plot is
able to convey these relationships clearly. In comparison, the mosaic
plot is not only confusing to interpret when there are more than 3
variables, but also fails in identifying any meaningful dependence
structure among these variables.

\begin{figure}
\centering
\includegraphics{Supplementary-Material_files/figure-latex/unnamed-chunk-38-1.pdf}
\caption{\label{fig:mp_sim_multi}Mosaic Plot for Hypothetical
Multivariate Ordinal Data after accounting for both \(G\) and \(Z\).}
\end{figure}

\begin{figure}
\includegraphics{Supplementary-Material_files/figure-latex/unnamed-chunk-39-1} \caption[BECCR Prediction Bubble Plot for Hypothetical Multivariate Ordinal Data after accounting for both $G$ and $Z$]{\label{fig:BECCR_sim_multi}BECCR Prediction Bubble Plot for Hypothetical Multivariate Ordinal Data after accounting for both $G$ and $Z$, with $Z=z_1$ being the top figure and $Z=z_2$ being the bottom figure.}\label{fig:unnamed-chunk-39}
\end{figure}

\subsubsection{Discussion}\label{discussion}

In summary, through the above simulations, we demonstrate how effective
the proposed BECCR Prediction Bubble Plots can be used to capture and
visualize various dependence structures --- no matter whether they are
linear, non-linear, different among different subgroups, and/or have
subgroups of different sizes --- in multi-dimensional contingency
tables. While one can use the usual mosaic plots to obtain similar
insights about those dependence structures --- when they are interpreted
correctly and when the dimension of the data is no more than three, some
might have a difficult time seeing those patterns easily and clearly,
compared to what they see in the corresponding BECCR Prediction Plots.
Not to mention that the BECCR Prediction Plots appear to have absolute
advantages over mosaic plots when the dimension of the data is greater
than three.

However, it's important to emphasize that our main goal here is
\(\underline{\text{not}}\) to show that BECCR Prediction Bubble Plots
``outperform'' the mosaic plots. Rather, we are proposing to use BECCR
Prediction Bubble Plots as a \emph{complementary} EDA tool and technique
for visualizing dependence structures in categorical data, on top of
mosaic plots and/or any existing visualization methods. One way to look
at the mutual complement between BECCR Prediction Bubble Plots and
mosaic plots for categorical data is to think of the role of
Least-Squares (LS) fitted lines to scatter plots for quantitative data.
No one would claim that LS-fitted lines are more important than scatter
plots when visualizing relationships between quantitative variables; we
use scatter plots to display observed data while using LS-fitted lines
to capture potential \emph{patterns} in quantitative data, to figure out
what we should do (which models we may want to fit) at the modeling step
of data analysis. Similarly, one can use mosaic plots to show the
observed counts/proportions of the data, while further using BECCR
Prediction Bubble Plots to help identify critical dependence structures
and potential interactions in categorical data before moving to fit a
model on those data. It's our firm belief that this newly developed
graphical tool can help us do a better job in modeling categorical data.

\newpage

\section{Real World Data Application}\label{real-world-data-application}

Chronic pain in adults occur more and more frequently as we age, and
this pain can often impact life or work activities. These kinds of
high-impact chronic pains are one of the most common reasons why adults
seek medical care, and the pain is associated with decreased quality of
life, opioid dependence, and poor mental health. As such, it is
important to study how different people are affected by pain, and what
factors might be associated with one's perception of pain is important
to study. As pain is a subjective measure, there could be tendencies
within certain subpopulations who may categorize their pain differently,
and we want to investigate what factors could impact this.

To accomplish this task, we used data from 2019 National Health
Interview Survey (NHIS) from the National Center for Health Statistics
in the CDC. We use the Sample Adult interview data\footnote{originally
  \(n=31997\) but after filtering out all missing data for our desired
  variables, we are left with an analytic sample \(n=19185\).} and will
use the following variables adapted from this data set:

\begin{itemize}
\tightlist
\item
  Amount of Pain, response variable \(Y\), which is an ordinal variable
  with 3 levels: Small Pain (\(y_1\) = SP), Moderate Pain (\(y_2\) =
  MP), and Great Pain (\(y_3\) = GP)
\item
  Frequency of Pain, explanatory variable \(X\), which is an ordinal
  variable with 3 levels: Low (\(x_1\)), Medium (\(x_2\) = Med), and
  High (\(x_3\))
\item
  Ethnicity, explanatory variable \(G\), which is a binary ordinal
  variable: People of the Global Majority (\(g_1\) = POGM) and White
  (\(g_2\))
\item
  Sex, explanatory variable \(Z\), which is a binary ordinal variable:
  Men (\(z_1\)) and Women (\(z_2\))
\end{itemize}

The actual data set is outlined in the Table \ref{tab:NHISpain}.

\begin{table}[H]

\caption{\label{tab:NHISpain}Perception of Amount of Pain
                    Suffered by Pain Frequency, Sex, and Ethnicity}
\centering
\begin{tabular}[t]{ccccc}
\toprule
\multicolumn{3}{c}{ } & \multicolumn{2}{c}{Ethnicity ($G$)} \\
\cmidrule(l{3pt}r{3pt}){4-5}
Pain Amount ($Y$) & Pain Frequency ($X$) & Sex ($Z$) & POGM ($g_1$) & White ($g_2$)\\
\midrule
 &  & Men & 911 & 2520\\
\cmidrule{3-5}
 & \multirow{-2}{*}{\centering\arraybackslash Low} & Women & 1126 & 2649\\
\cmidrule{2-5}
 &  & Men & 58 & 264\\
\cmidrule{3-5}
 & \multirow{-2}{*}{\centering\arraybackslash Medium (Med)} & Women & 54 & 221\\
\cmidrule{2-5}
 &  & Men & 54 & 346\\
\cmidrule{3-5}
\multirow{-6}{*}{\centering\arraybackslash Small Pain (SP)} & \multirow{-2}{*}{\centering\arraybackslash High} & Women & 75 & 232\\
\cmidrule{1-5}
 &  & Men & 481 & 1082\\
\cmidrule{3-5}
 & \multirow{-2}{*}{\centering\arraybackslash Low} & Women & 720 & 1427\\
\cmidrule{2-5}
 &  & Men & 120 & 441\\
\cmidrule{3-5}
 & \multirow{-2}{*}{\centering\arraybackslash Medium (Med)} & Women & 212 & 663\\
\cmidrule{2-5}
 &  & Men & 170 & 708\\
\cmidrule{3-5}
\multirow{-6}{*}{\centering\arraybackslash Moderate Pain (MP)} & \multirow{-2}{*}{\centering\arraybackslash High} & Women & 231 & 859\\
\cmidrule{1-5}
 &  & Men & 134 & 293\\
\cmidrule{3-5}
 & \multirow{-2}{*}{\centering\arraybackslash Low} & Women & 266 & 440\\
\cmidrule{2-5}
 &  & Men & 69 & 132\\
\cmidrule{3-5}
 & \multirow{-2}{*}{\centering\arraybackslash Medium (Med)} & Women & 125 & 236\\
\cmidrule{2-5}
 &  & Men & 189 & 523\\
\cmidrule{3-5}
\multirow{-6}{*}{\centering\arraybackslash Great Pain (GP)} & \multirow{-2}{*}{\centering\arraybackslash High} & Women & 337 & 817\\
\bottomrule
\end{tabular}
\end{table}

Given this data set, we want to explore if it has any peculiar features
and if we can glean any insights about the relationships about the
variables. For each variable in the data, the table consisting of the
proportion of observations in each category is listed in the Appendix
\ref{A}.

Besides using those tables as a numerical summary of the data, we also
want to utilize graphs to visualize as part of our EDA of the data set.
The first visualization tool we use is the mosaic plot, shown as Figure
\ref{fig:mp_NHIS}.

\begin{figure}
\centering
\includegraphics{Supplementary-Material_files/figure-latex/unnamed-chunk-41-1.pdf}
\caption{\label{fig:mp_NHIS}Mosaic Plot of Perception of Pain Adults
Feel by Pain Frequency, Sex, and Ethnicity.}
\end{figure}

From this mosaic plot, we can see that for every response of pain level,
there doesn't seem to be much of a difference in the proportions of men
and women responding based off their ethnicity. That is, the proportions
of men and women for POGM who respond to each pain level seem to be
proportionally similar to those of white men and women. Further, the
relative differences in the proportions between the pain frequencies
also seem similar between POGM and white participants.

However, it is unclear whether or not there are tangible interactions
between the variables and what their relationship would be. To hopefully
gain more useful insight into this, we create a BECCR Prediction Bubble
Plot shown in Figure \ref{fig:BECCR_NHIS} to help.

\begin{figure}
\includegraphics{Supplementary-Material_files/figure-latex/unnamed-chunk-43-1} \caption[Prediction Bubble Plots of 2019 NHIS Adult Survey Data about Amount of Pain received by Reported Pain Frequency, Sex, and Ethnicity.]{\label{fig:BECCR_NHIS}BECCR Prediction Bubble Plots of 2019 NHIS Adult Survey Data about Amount of Pain received by Reported Pain Frequency, Sex, and Ethnicity. (Top) Combination of categories (POGM, Sex, Pain Frequency) and (Bottom) combination of categories (White, Sex, Pain Frequency) estimated by copula regression in the 1000 bootstrap resampling}\label{fig:unnamed-chunk-43}
\end{figure}

Looking at Figure \ref{fig:BECCR_NHIS}, one can easily see some clear
relationships arise from the BECCR predicted categories of the amount of
pain. For both POGM and White Men, they seem to share a similar clear
positive relationship between pain frequency (\(X\)) and amount of pain
perceived (\(Y\)). This relationship seems to be quite strong as the
prediction jumps from Small Pain (SP) to Great Pain (GP) after
increasing from low pain frequency to medium pain frequency. However,
the results for Women tell a different story, depending on whether they
are POGM or White. Those different stories imply that there is a
potential interaction effect between Ethnicity (\(G\)) and Pain
Frequency (\(X\)).

Similar to White Men, White Women also seem to have a positive
relationship between Pain Frequency (\(X\)) and Pain Percieved (\(Y\)),
but this relationship seems more linear than White Men's as we predict
that White Women feel moderate pain (MP) when they have medium (med)
pain frequency, instead of jumping directly up to great pain (GP) like
it was for the White Men. On the other hand, for POGM Women, the amount
of pain seems to be consistently high, even when the pain frequency is
low. Though it dips back to moderate pain (MP) when the pain frequency
is medium (med), it goes back to great pain (GP) when the pain frequency
is high. This relationship is unlike the relationships we observe for
POGM Men, as this seems quadratic in a sense. All those differences tell
us that Sex (\(Z\)) and Ethnicity (\(G\)) might interact with each
other.

Not only does this suggest that there may be differences in how one
should interpret the pain data for men and women, but we also have to
take into consideration their ethnicity as well; in other words, it's
important to pay attention to the
\(\underline{\text{intersectionality}}\) between Sex (\(Z\)) and
Ethnicity (\(G\)) when predicting a patient's pain level by their pain
frequency (\(X\)).

To illustrate how ignoring intersectionality may lead Simpson's paradox,
let us first consider the BECCR Prediction Bubble Plot
\(\underline{\text{without}}\) considering their Ethnicity (\(Z\)). This
is illustrated in Figure \ref{fig:BECCR_NHIS_noZ}. Taking a look at this
plot, we can see that the pattern between the reported pain amount and
reported pain frequency is the same strong positive relationship that we
saw in Figure \ref{fig:BECCR_NHIS} for both White and POGM men. When you
look at women without considering their ethnicity, we can see that the
aggregated plot (the right half of Figure\ref{fig:BECCR_NHIS_noZ}) looks
somehow similar to the BECCR Prediction Bubble Plot of POGM women (the
right half of the top plot in Figure \ref{fig:BECCR_NHIS}), with the
White Women's positive linear relationship (``the right half of the
bottom plot in Figure \ref{fig:BECCR_NHIS}) being hidden. This is an
effect of amalgamating our subpopulations together, and is what we are
calling''Simpson's Paradox'' for response variables with more than 2
categories in this paper. As a consequence, one may over-estimate White
Women's pain level when they report low pain frequency and prescribe too
strong pain medicine for those White Women.

We can take this another step further and consider the BECCR Prediction
Bubble Plot of the NHIS data without considering both Ethnicity (\(G\))
\emph{and} Sex (\(Z\)), which results in Figure
\ref{fig:BECCR_NHIS_noZ_noG}.

\begin{figure}
\centering
\includegraphics{Supplementary-Material_files/figure-latex/unnamed-chunk-44-1.pdf}
\caption{\label{fig:BECCR_NHIS_noZ}BECCR Prediction Bubble Plots of 2019
NHIS Adult Survey Data about Amount of Pain received by Reported Pain
Frequency and Sex. Predicted category estimated by copula regression in
the 1000 bootstrap resamples.}
\end{figure}

\begin{figure}
\centering
\includegraphics{Supplementary-Material_files/figure-latex/unnamed-chunk-46-1.pdf}
\caption{\label{fig:BECCR_NHIS_noZ_noG}BECCR Prediction Bubble Plots of
2019 NHIS Adult Survey Data about Amount of Pain received by Reported
Pain Frequency. Predicted category estimated by copula regression in the
1000 bootstrap resamples.}
\end{figure}

\newpage

\section{\texorpdfstring{Appendix \label{A}}{Appendix }}\label{appendix}

\subsection{EDA for Single Variables for NHIS Adult Pain
Data}\label{NHIS_EDA}

\begin{verbatim}
## G
##      POGM     White 
## 0.2779255 0.7220745
\end{verbatim}

\begin{verbatim}
## Z
##       Men     Women 
## 0.4427938 0.5572062
\end{verbatim}

\begin{verbatim}
## X
##       Low       Med      High 
## 0.6280427 0.1352619 0.2366953
\end{verbatim}

\begin{verbatim}
## Y
##        SP        MP        GP 
## 0.4435757 0.3708105 0.1856138
\end{verbatim}

\end{document}
